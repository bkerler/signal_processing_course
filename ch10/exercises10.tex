\newpage
\section{Exercises: Linear Time-invariant Systems}
\begin{enumerate}
\item The impulse response of an LTI system is defined as:
\begin{equation}
    h(t) = \frac{42\sin(\omega_c t)}{\pi t}.
    \label{eq:ir_42rect}
\end{equation}
The input to the LTI system is a periodic signal with a fundamental period $T=1$:
\begin{equation}
    x(t) = \sum_{n=-\infty}^{\infty}\delta(t-n).
\end{equation}
When the signal $x(t)$ is fed into an LTI system, the output signal is given by a convolution of the input signal with the output signal:
\begin{equation}
    y(t) = \int_{-\infty}^{\infty} h(t-\tau)x(\tau) d\tau.
\end{equation}
\begin{enumerate}[a)]
\item Determine the Fourier transform $\hat{x}(\omega)$ of the input signal. Plot $\hat{x}(\omega)$ over the range of angular frequencies $-7\pi < \omega < 7\pi$. Hint: It might help if you start by expressing the signal $x(t)$ using a Fourier series representation first, and then apply Equation \ref{eq:fsftgen}.
\item Determine the frequency response $\mathcal{H}(\omega)$ of the LTI system. In other words, Fourier transform the impulse response signal given in Equation \ref{eq:ir_42rect}. It should be a fairly simple function. Make a plot of $|\mathcal{H}(\omega)|$ on the same graph as $\hat{x}(\omega)$ using $\omega_c = 5\pi$.
\item Use your plot in b) to determine $y(t)$, the output of the LTI system characterized using $h(t)$ when $\omega_c=5\pi$. Hint: use property that convolution in time domain is multiplication in frequency domain.
\item What values of $\omega_c$ will result in a non-zero constant output $y(t)=c$. What is the constant $c$?
\end{enumerate}

\item A running average system is defined as:
  \begin{equation}
    y[n] = \mathcal{T}\{x[n]\} = \frac{1}{L}\sum_{k=0}^{L-1} x[n-k]
  \end{equation}

  \begin{enumerate}[a)]
    \item Show that the running average system is a linear time
      invariant system using the test for linearity and time
      invariance.
    \item What is the impulse response $h[n] =
      \mathcal{T}\{\delta[n]\}$ of the running average system?
    \item How many non-zero values does the impulse response $h[n]$ have?
    \item We feed a discrete-time complex sinusoidal signal
      $x[n]=e^{i\hat{\omega}_0 n}$ into the system. Show that the
      output is of the form $y[n]=A e^{i\phi} e^{i\hat{\omega}_0 n}$,
      in other words, a discrete-time complex sinusoidal signal with
      the same frequency as the input signal.
      \item Continue with task d). Let us assume that
        $L=4$. What is the amplitude of the output signal $A$ when we
        have a normalized angular frequency of:
        \begin{enumerate}
        \item $\hat{\omega}_0 = 0$ radians per sample?
        \item $\hat{\omega}_0 = \pi$ radians per sample?
        \item $\hat{\omega}_0 = 0.5\pi$ radians per sample?          
        \end{enumerate}
      
    \item Show that system $y_2[n]=\mathcal{T}\{ \mathcal{T}\{x[n]\} \}$ is also an LTI system.
    \item What is the impulse response $h_2[n]=\mathcal{T}\{\mathcal{T}\{\delta[n]\}\}$? Sketch a plot of the non-zero values $h_2[n]$.
  \end{enumerate}
  
  \item Obtain the demonstration code that implements a reverb effect
    using a convolution operation:
    \begin{equation}
      y[n]=h[n]*x[n],
      \end{equation}
    where $x[n]$ is the input signal, $h[n]$ is the impulse response
    of a room, and $y[n]$ is the output signal with the reverb
    effect. Download \verb|7na.wav| from the course GitHub examples,
    \url{https://github.com/jvierine/signal_processing/tree/master/018_reverb}
    or alternatively use your own audio file.

\begin{enumerate}[a)]

\item Run the example code and verify that the filtered signal indeed sounds like it
is played in a large room by playing \verb|7na.wav| and the file \verb|reverb.wav|
produced by the script.

\item Find the part in the code where a convolution between the audio
  signal and the impulse response is evaluated. Plot the impulse
  response of the FIR filter applied to the input signal.

\item Figure out how to increase and reduce the amount of reverb, i.e.,
to make it sound like the audio signal is played in a large or small
room with many surfaces that reflect sound waves. What does the
impulse response for a large room and a small room look like?

\item Now try to create an impulse response of an echo from a distance
  of 1000 meters using the following impulse response
  $h[n]=\delta[n]+0.5\delta[n-n_0]$. Assuming that the propagation velocity of
  sound is 343 $\frac{\mathrm{m}}{\mathrm{s}}$, determine a suitable value for $n_0$.

\item Implement several length running average filters to
  the audio signal (e.g., Equation \ref{eq:running_mean}), which
  average together 4, 8, 16, and 100 samples. Can you detect
  with your ear what happens to the low and high frequency components
  of the signal as a result of the operation?

\item Come up with your own model of the acoustics of some space. Try to figure
out at what propagation delays and amplitudes you expect to receive scattered sound
in that space. Implement it and try out how it sounds like!
\end{enumerate}


\item You can use the sound card on your computer to implement a
  sonar\sidenote{This last exercise is technically a bit more
    challenging, so this is only intended for those who want to
    explore practical applications. There is a good chance that you
    will need to do a lot of trial and error testing before you get
    this to work. I will try to demonstrate this in class.}. All you need to do is to emit an audio signal
  consisting of a train of short pulses (Dirac comb) with a speaker
  connected to your computer. By using a suitable pulse spacing (e.g.,
  0.2 seconds), you can probe your surroundings for acoustic
  scattering. Use the audio example programs to create an audio file
  that you can play with your computer.

Record sounds with a microphone with a program such as audacity
simultaneously while you play the audio signal with pulses. Analyze
the resulting audio file and determine the impulse response of your
surroundings.

You can use a large flat surface (e.g., a piece of cardboard) as an
acoustic reflector to test your sonar. You can move the piece of
cardboard closer or further away from your computer to test if a
spike-like feature of the impulse response corresponds to your
acoustic reflector.


\end{enumerate}