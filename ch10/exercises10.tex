\newpage
\section{Exercises: Linear Time-invariant Systems}
\begin{enumerate}

\item Prove that the convolution operation is commutative. That is, show that $a[n]*b[n]=b[n]*a[n]$. In addition, show that this holds true for the continuous-time convolution operation. 

\item A running average system is defined as:
  \begin{equation}
    y[n] = \mathcal{T}\{x[n]\} = \frac{1}{L}\sum_{k=0}^{L-1} x[n-k]
  \end{equation}

  \begin{enumerate}[a)]
    \item Show that the running average system is a linear time
      invariant system using the test for linearity and time
      invariance.
    \item What is the impulse response $h[n] =
      \mathcal{T}\{\delta[n]\}$ of the running average system?
    \item How many non-zero values does the impulse response $h[n]$ have?
    \item We feed a discrete-time complex sinusoidal signal
      $x[n]=e^{i\hat{\omega}_0 n}$ into the system. Show that the
      output is of the form $y[n]=A e^{i\phi} e^{i\hat{\omega}_0 n}$,
      in other words, a discrete-time complex sinusoidal signal with
      the same frequency as the input signal.
      \item Continue with task d). Let us assume that
        $L=4$. What is the amplitude of the output signal $A$ when we
        have a normalized angular frequency of:
        \begin{enumerate}
        \item $\hat{\omega}_0 = 0$ radians per sample?
        \item $\hat{\omega}_0 = \pi$ radians per sample?
        \item $\hat{\omega}_0 = 0.5\pi$ radians per sample?          
        \end{enumerate}
      
    \item Show that system $y_2[n]=\mathcal{T}\{ \mathcal{T}\{x[n]\} \}$ is also an LTI system.
    \item What is the impulse response $h_2[n]=\mathcal{T}\{\mathcal{T}\{\delta[n]\}\}$? Sketch a plot of the non-zero values $h_2[n]$.
  \end{enumerate}
  
  \item Obtain the demonstration code that implements a reverb effect
    using a convolution operation:
    \begin{equation}
      y[n]=h[n]*x[n],
      \end{equation}
    where $x[n]$ is the input signal, $h[n]$ is the impulse response
    of a room, and $y[n]$ is the output signal with the reverb
    effect. Download \verb|7na.wav| from the course GitHub examples,
    \url{https://github.com/jvierine/signal_processing/tree/master/018_reverb}
    or alternatively use your own audio file.

\begin{enumerate}[a)]

\item Run the example code and verify that the filtered signal indeed sounds like it
is played in a large room by playing \verb|7na.wav| and the file \verb|reverb.wav|
produced by the script.

\item Find the part in the code where a convolution between the audio
  signal and the impulse response is evaluated. Plot the impulse
  response of the FIR filter applied to the input signal.

\item Figure out how to increase and reduce the amount of reverb, i.e.,
to make it sound like the audio signal is played in a large or small
room with many surfaces that reflect sound waves. What does the
impulse response for a large room and a small room look like?

\item Now try to create an impulse response of an echo from a distance
  of 1000 meters using the following impulse response
  $h[n]=\delta[n]+0.5\delta[n-n_0]$. Assuming that the propagation velocity of
  sound is 343 $\frac{\mathrm{m}}{\mathrm{s}}$, determine a suitable value for $n_0$.

\item Implement several length running average filters to
  the audio signal (e.g., Equation \ref{eq:running_mean}), which
  average together 4, 8, 16, and 100 samples. Can you detect
  with your ear what happens to the low and high frequency components
  of the signal as a result of the operation?
\end{enumerate}


\end{enumerate}