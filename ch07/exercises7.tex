\newpage
\section{Exercises: Fourier Series}

\begin{enumerate}
  % Exercise 1
  \item A signal is defined as:
        \begin{equation}
          x(t) = 7 \sin(3 \pi t + 0.2\pi) + 3 \cos(7 \pi t + 0.5\pi)
        \end{equation}
        The independent variable $t$ is in units of seconds.
        \begin{enumerate}[a)]
          % Exercise 1a)
          \item Express this signal in the form of Equation \ref{eq:general_spectrum}. What are the values of the coefficients $c_k \in \mathbb{C}$ and angular frequencies $\omega_k \in \mathbb{R}$?
          % Exercise 1b)
          \item The coefficients $c_{k}$, satisfy $c_{-k}=c_{k}^{*}$, why?
          % Exercise 1c)
          \item Show that this signal is periodic by showing that the frequencies of the individual frequency components are commensurable.
          % Exercise 1d)
          \item Use Euclid's algorithm to find the fundamental angular frequency $\omega$. 
          What is the fundamental frequency in units of hertz and rad/s?
          % Exercise 1e)
          \item What is the fundamental period $T$ of this signal in seconds?
          % Exercise 1f)
          \item We delay the signal $x(t)$ by 0.5 seconds $y(t) = x(t-\frac{1}{2})$. 
          Write $y(t)$ in the following format:
                \begin{equation}
                  y(t) = 7 \sin(3\pi t + \phi_0) + 3 \cos(7\pi t + \phi_1)
                \end{equation}
                What are the values $\phi_0$ and $\phi_1$?
          % Exercise 1g)
          \item We create a new signal $z(t) = x(t) + e^{i \sqrt{2} t + 13}$. 
          Why is the signal $z(t)$ no longer periodic with a finite period? Is the signal $z(t)$ real-valued?
        \end{enumerate}

  % Exercise 2
  \item A signal is defined as:
        \begin{equation}
          x(t) = e^{-i (6\pi  t + 0.3)}  + 4e^{i (60\pi  t + 0.42)} + 4e^{-i (60\pi t + 0.42)}  + e^{i (6\pi t + 0.3)}
        \end{equation}
        The independent variable $t$ is in units of seconds.
        \begin{enumerate}[a)]
          % Exercise 2a)
          \item Why is this signal real-valued?
          % Exercise 2b)
          \item Is this signal periodic? If so, what is the fundamental angular frequency $\omega$ and the fundamental period $T$?
        \end{enumerate}

  % Exercise 3
  \item The Fourier series coefficients for a pulsed signal $x(t)$ with
        a fundamental period $T=1$ is described as:
        \begin{equation}
          c_k = \left\{\begin{array}{ccc}
            \frac{1}{10}                                                               & \mathrm{when}      & k=0 \\
            \frac{1}{\pi k}e^{-i\frac{\pi}{10}  k }\sin\left(\frac{\pi}{10} k  \right) & \mathrm{otherwise}
          \end{array}
          \right.
        \end{equation}
        See Equations \ref{eq:pulsecoeff0} and \ref{eq:pulsecoeff1} for the derivation of the Fourier coefficients.
        \begin{enumerate}[a)]
          % Exercise 3a)
          \item Plot the partial sum $x_N(t)$ of the Fourier series using $N=101$. You can use Listing \ref{lst:fourier_square_code} to help you out.
          % Exercise 3b)
          \item What are the Fourier series coefficients for a signal delayed in time by $0.2$ seconds. Plot the partial sum to verify.
          % Exercise 3c)
          \item What are the Fourier series coefficients for $y(t)=\frac{d}{dt}x(t)$? Make a plot to verify.
        \end{enumerate}

\end{enumerate}

