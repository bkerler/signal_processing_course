\newpage
\section{Exercises: Fourier Series}

\begin{enumerate}
\item A signal is defined as:
\begin{equation}
    x(t) = 7 \sin(3 \pi t + 0.2\pi) + 3 \cos(7 \pi t + 0.5\pi)  
\end{equation}
The independent variable $t$ is in units of seconds. 
\begin{enumerate}[a)]
    \item Express this signal in the form of Equation \ref{eq:general_spectrum}. What are the values of the coefficients $c_k \in \mathbb{C}$ and angular frequencies $\omega_k \in \mathbb{R}$?
    \item The coefficients $c_{k}$, satisfy $c_{-k}=c_{k}^{*}$, why?
    \item Show that this signal is periodic by showing that the frequencies of the individual frequency components are commensurable.
    \item Use Euclid's algorithm to find the fundamental angular frequency $\omega$. What is the fundamental frequency in units of hertz and rad/s?
    \item What is the fundamental period $T$ of this signal in seconds?
    \item We delay the signal $x(t)$ by 0.5 seconds $y(t) = x(t-\frac{1}{2})$. Write $y(t)$ in the following format:
\begin{equation}
    y(t) = 7 \sin(3\pi t + \phi_0) + 3 \cos(7\pi t + \phi_1)
\end{equation}
What are the values $\phi_0$ and $\phi_1$?
\item We create a new signal $z(t) = x(t) + e^{i \sqrt{2} t + 13}$. Why is the signal $z(t)$ no longer periodic with a finite period? Is the signal $z(t)$ real-valued?
\end{enumerate}

\item A signal is defined as:
  \begin{equation}
    x(t) = e^{-i (6\pi  t + 0.3)}  + 4e^{i (60\pi  t + 0.42)} + 4e^{-i (60\pi t + 0.42)}  + e^{i (6\pi t + 0.3)} 
  \end{equation}
  The independent variable $t$ is in units of seconds. 
  \begin{enumerate}[a)]
%  \item Make a plot of the frequency components of the signal, similar to the one in Figure \ref{fig:exspecsin}.
  \item Why is this signal real-valued?
    \item Is this signal periodic? If so, what is the fundamental angular frequency $\omega$ and the fundamental period $T$?
   \end{enumerate}


\item The Fourier series coefficients for a pulsed signal $x(t)$ with
  a fundamental period $T=1$ is described as:
  \begin{equation}
    c_k = \left\{\begin{array}{ccc}
    \frac{1}{10} & \mathrm{when} & k=0 \\
    \frac{1}{\pi k}e^{-i\frac{\pi}{10}  k }\sin\left(\frac{\pi}{10} k  \right) & \mathrm{otherwise}
    \end{array}
    \right.
  \end{equation}
  See Equations \ref{eq:pulsecoeff0} and \ref{eq:pulsecoeff1} for the derivation of the Fourier coefficients.
  \begin{enumerate}[a)]
  \item Plot the partial sum $x_N(t)$ of the Fourier series using $N=101$. You can use Listing \ref{lst:fourier_square_code} to help you out.
  \item What are the Fourier series coefficients for a signal delayed in time by $0.2$ seconds. Plot the partial sum to verify.
  \item What are the Fourier series coefficients for $y(t)=\frac{d}{dt}x(t)$? Make a plot to verify.
  \end{enumerate}
  
\end{enumerate}


\if 0
\section{Programming Exercise 1}

\begin{itemize}
\item[a)] 
Using Equation \ref{eq:dirac_comb_coeff}, implement a program that
calculates a partial sum approximation of a Dirac comb with
period of $T=0.5$ seconds. Use $N=50$ as the number of complex
exponential signals to include in the partial sum. Evaluate the signal
from t=0 to t=4 seconds at 10 kHz sample rate. Here's some partial
code, which almost does the job.
\begin{lstlisting}[language=Python,numbers=none]
# define the sample rate (Hz)
sample_rate=10000.0
# create time array 0 to 4 seconds
t=n.arange(4.0*sample_rate)/sample_rate
# initialize empty vector to hold Fourier series
sig=n.zeros(len(t),dtype=n.complex64)
N=50
for k in range(-N,N):
    sig+=...# complete this line
    
plt.plot(sig.real)
plt.plot(sig.imag)
plt.show()
\end{lstlisting}

\item[b)] Delay the signal by 0.1 seconds by applying an adjustment to the complex coefficients $c_k$. Use Equation \ref{eq:time_shift_phasor}. Plot the non-delayed and delayed signals in the same plot and verify that the signal is delayed by the right amount.
\end{itemize}


\fi
