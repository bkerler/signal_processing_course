\newpage
\section{Suggested solutions: Fourier Series}

\begin{enumerate}
\item Let $x(t)$ be the signal:
$$x(t)=7\sin(3\pi t+0.2\pi)+3\cos(7\pi t+0.5\pi).$$

\begin{enumerate}[a)]
\item Have that:
\begin{align*}
    \sin\theta&=\frac{1}{2i}(e^{i\theta}-e^{-i\theta}), \\
    \cos\theta&=\frac{1}{2}(e^{i\theta}+e^{-i\theta}).
\end{align*}
This gives:
\begin{align*}
    x(t)&=\frac{7}{2i}(e^{i(3\pi t+0.2\pi)}-e^{-i(3\pi t+0.2\pi)})+ \frac{3}{2}(e^{i(7\pi t+0.5\pi)}+e^{-i(7\pi t+0.5\pi)}), \\
    &=\left(\frac{7}{2i}e^{i0.2\pi}\right)e^{i3\pi t}-\left(\frac{7}{2i}e^{-i0.2\pi}\right)e^{-i3\pi t} + \left(\frac{3}{2}e^{i0.5\pi}\right)e^{i7\pi t}+\left(\frac{3}{2}e^{-i0.5\pi}\right)e^{-i7\pi t},
\end{align*}
giving:
$$c_{k}=\left(\frac{3}{2}e^{-i0.5\pi},-\frac{7}{2i}e^{-i0.2\pi},\frac{7}{2i}e^{i0.2\pi},\frac{3}{2}e^{i0.5\pi}\right),$$
for which the corresponding angular frequencies are:
$$\omega_{k}=(-7\pi,-3\pi,3\pi,7\pi).$$

\item For any real signal, the Fourier coefficients satisfy $c_{-k}=c_{k}^{*}$ and $x(t)$ is real. 

\item The signal is periodic if $\omega_{i}/\omega_{j}\in\mathbb{Q}$ for every pair $i,j\in \{-7,-3,3,7\}$ with $i\neq j$. In this case every $\omega_{i}$ is an integer multiply of $\pi$, so every ratio is a rational number as all the $\pi$s cancel.

\item Using Euclid's algorithm we have:
\begin{align*}
    &(3\pi,7\pi), \\
    &(3\pi,7\pi-3\pi), \\
    &(3\pi,4\pi), \\
    &(3\pi,4\pi-3\pi), \\
    &(3\pi,\pi), \\
    &(3\pi-\pi,\pi), \\
    &(2\pi,\pi), \\
    &(2\pi-\pi,\pi), \\
    &(\pi,\pi).
\end{align*}
Hence, the fundamental angular frequency is $\omega=\pi\ \text{rad/s}$. In units of hertz, we have that $\omega=2\pi f$, so
$$f=\frac{\omega}{2\pi}=\frac{\pi}{2\pi}=\frac{1}{2}.$$
Thus, the fundamental frequency is $f=1/2\ \text{Hz}$. 

\item If the fundamental angular frequency is $\omega=\pi$, then the fundamental period is related by $T=2\pi/\omega$, hence
$$T=\frac{2\pi}{\omega}=\frac{2\pi}{\pi}=2\ \text{seconds}.$$

\item Define a new signal $y(t)=x(t-\frac{1}{2})$. That is, we delay the signal $x(t)$ by $\frac{1}{2}$. We get:
$$y(t)=7\sin(3\pi \left(t-\frac{1}{2}\right)+0.2\pi)+3\cos(7\pi \left(t-\frac{1}{2}\right)+0.5\pi).$$
Simplifying gives:
$$y(t)=7\sin(3\pi t-1.3\pi)+3\cos(7\pi t-3\pi).$$
Hence, $\phi_{0}=-1.3\pi$ and $\phi_{1}=-3\pi$. 

\item Define another signal $z(t)=x(t)+e^{i\sqrt{2}+13}$. This new signal is not commensurable since we have a factor of $\sqrt{2}$, the signal is also not real-valued since we don't have a corresponding complex conjugate pair. 
\end{enumerate}

\item Let $x(t)$ be a signal defined as:
$$x(t)=e^{-i(6\pi t+0.3)}+4e^{i(60\pi t+0.42)}+4e^{-i(60\pi t+0.42)}+e^{i(6\pi t+0.3)},$$
where $t$ is measured in seconds. 
\begin{enumerate}[a)]
\item This signal is real-valued since we have two pairs, each with the same frequency, but different signs. Thus, the signal can be written as a real signal for which it takes the following form:
$$x(t)=8\cos(60\pi t+0.42)+2\cos(6\pi t+0.3).$$
\item In this case, we have the angular frequencies of $\omega_{1}=60\pi$ and $\omega_{2}=6\pi$ for which $\omega_{1}/\omega_{2}=60\pi/6\pi=10$. We get a rational number, so the signal is periodic. 
\item To find the fundamental angular frequency, we use Euclid's algorithm. This gives:
\begin{align*}
    &(60\pi,6\pi), \\
    &(54\pi,6\pi), \\
    &(48\pi,6\pi), \\
    &(42\pi,6\pi), \\
    &(36\pi,6\pi), \\
    &(30\pi,6\pi), \\
    &(24\pi,6\pi), \\
    &(18\pi,6\pi), \\
    &(12\pi,6\pi), \\
    &(6\pi,6\pi), 
\end{align*}
therefore, the fundamental angular frequency is $6\pi$ in units of radians per second. 

\item The fundamental period is:
$$T=\frac{2\pi}{\omega}=\frac{2\pi}{6\pi}=\frac{1}{3},$$
hence $T=\frac{1}{3}$ in units of seconds.
\end{enumerate}

\item Let the Fourier series coefficients of a periodic signal $x(t)$ with fundamental period $T=1$ be:
$$c_{k}=\begin{cases}
    \frac{1}{10}, \hspace{2.3cm}\quad k=0, \\
    \frac{1}{\pi k}e^{-i\frac{\pi}{10}k}\sin(\frac{\pi}{10}k),\quad \text{otherwise}.
\end{cases}$$

\begin{enumerate}[a)]
\item Using Python, we can implement a simple program for the partial sum with $N=101$. Listing \ref{code:7.3a} shows a way. 
\lstinputlisting[language=Python,caption=Suggested solution to a),label=code:7.3a]{ch07/code/ex7_3a.py}

\item The Fourier coefficients of a delayed signal are related to the undelayed signal by $c_{k}'=e^{-i\frac{2\pi k\tau}{T}}c_{k}$. 
The script in Listing \ref{code:7.3a} can be modified to account for delay. The delayed version is shown in Listing \ref{code:7.3b}.
\lstinputlisting[language=Python,caption=Suggested solution to b),label=code:7.3b]{ch07/code/ex7_3b.py}

\item Define $y(t)=\frac{d}{dt}x(t)$. If $c_{k}$ are the Fourier coefficients for $x(t)$, then the Fourier coefficients for $y(t)$ is:
$$d_{k}(t)=i \frac{2\pi k}{T}c_{k}.$$
Using Python we have Listing \ref{code:7.3c}.
\lstinputlisting[language=Python,caption=Suggested solution to c),label=code:7.3c]{ch07/code/ex7_3c.py}

\end{enumerate}


\end{enumerate}