\section{The Decibel Scale}
If you do enough signal processing, you will eventually encounter the
\index{decibel}{decibel} scale\sidenote{The decibel scale is widely used in
engineering. The ``bel'' part is named after the inventor of the
telephone, Alexander Graham Bell. Why the unit is bel instead of bell
is not known to me.}, which is a logarithmic scale used for numerical
quantities that denote \emph{power}. This scale is defined as
\begin{equation}
\boxed{
P_{\mathrm{dB}} = 10 \log_{10}(P/P_0),
}
\end{equation}
where $P_0$ is a reference power level that a power $P$ is compared
against. This reference power level $P_0$ is usually something
physically meaningful, such as 1 watt of power, or the smallest
audible signal power that a human can discern.

When a quantity is expressed in the decibel scale, it is typically
prefixed with the abbreviation ``dB''. The main reason why the decibel
scale is useful is that it is logarithmic. This makes it easy to
numerically or visually represent a quantity that has a
large \emph{\index{dynamic range}{dynamic range}}. The dynamic range
means the ratio of the largest and smallest value of a quantity that
is meaningful for investigating that quantity.
\begin{margintable}
\begin{center}
\begin{tabular}{c|c}
dB & linear \\
\hline
\hline
0 & 1 \\
3 & $\approx$2 \\
6 & $\approx$4 \\
9 & $\approx$8 \\
-3 & $\approx$1/2 \\
-6 & $\approx$1/4 \\
-9 & $\approx$1/8 \\
-40 & 0.0001 \\
-30 & 0.001 \\
-20 & 0.01 \\
-10 & 0.1 \\
0 & 1 \\
10 & 10 \\
20 & 100 \\
30 & 1000 \\
40 & 10000 
\end{tabular}
\end{center}
\caption{Here is a table of commonly encountered decibel values in linear scale.}
\end{margintable}


Here are some examples of dynamic range. Humans are very good at
perceiving a relative difference in a quantity only if it has a small
dynamic range. It is very easy to judge that an apple costing \$2 is
twice as expensive as an apple costing \$1. It is not very easy for us
to perceive the relative difference in the weight of a grain of sand
(0.000004 kg) and the weight of Earth (5972000000000000000000000
kg). The first example has a small dynamic range, and the second
example has a large dynamic range.

If you are a scientifically thinking person, you will have probably
already counted the number of zeros in the weight of a grain of sand
and the weight of Earth, so you can mentally express these numbers in
the form $4 \cdot 10^{-6}$~kg and $5.972 \cdot 10^{24}$~kg. What you
are doing is actually converting the numbers into a logarithmic form,
where the exponents (-6 and 24) tell you the order of magnitude of
these numbers. And now by subtracting -6 from 24, you can tell that
the Earth's weight corresponds to approximately $10^{30}$ grains of
sand. This is essentially the idea behind the decibel scale, too. 


\newthought{Here's an example of using decibels in radio engineering}. Let us assume
that the electric potential of a sinusoidal signal is $A=0.316$ volts,
with the signal represented as:
\begin{equation}
U(t) = A \sin(t).
\label{eq:dbmsignal}
\end{equation}
Let us assume that this signal is terminated by a $R=50$ $\Omega$
load. This means that the average power fed into the load is:
\begin{equation}
P = \frac{A^2}{2\pi R}\int_0^{2\pi} \sin^2(t) dt = \frac{A^2}{2 R} \approx 10^{-3}~\mathrm{watts}.
\end{equation}
We arrived at this using Kirchoff's circuital laws $U=RI$ and $P=UI$,
integrating over one cycle of the time harmonic signal, and assuming
that the frequency of the signal is fairly small.

\begin{marginfigure}
\begin{circuitikz}
%\draw (0,0) to[sinusoidal voltage source=$U(t)=A\sin(t)$] (0,4) -- (4,4) to[R=$50~\Omega$] (4,0) -- (0,0);
\draw (0,0) to[sinusoidal voltage source] (0,4) -- (4,4) to[R=$50~\Omega$] (4,0) -- (0,0);
\end{circuitikz}
%\caption{Simple circuit.}
\end{marginfigure}

A typical reference power level in radio engineering is one milliwatt
of power, or $P_0 = 10^{-3}$ watts. In this case, the
acronym \index{dBm}{dBm} is typically used to denote the signal in
decibel scale, relative to one milliwatt. In this scale, our example
signal given in Equation \ref{eq:dbmsignal} with an amplitude of
$A=0.316$ volts would correspond to 0 dBm of power. The EISCAT
incoherent scatter radar, transmitting 1 MW of power, would be 90 dBm
in this scale.




