\newpage
\section{Audio Compression Example [Optional]}
In this application example, I'll show you how to estimate the
spectral components of an audio signal and how this can be used for
audio compression. I won't go very deep into either of these
topics. The main idea is to expose you to these concepts.

\newthought{Example: audio compression algorithms often use a sparse spectral
representation of an audio signal}.  The sounds produced by musical
instruments when playing a single note are often good examples of
periodic signals that occur in our daily life. Of course, real signals
are typically not exactly periodic, but for short intervals this
approximation is good.

A good example of this is the signal shown in
Figure \ref{fig:audio_spec}, which depicts the sound signal (instantaneous
relative air pressure at the microphone) emitted by a guitar when one
string is plucked.

A Fourier series approximation of an audio signal is often used in
audio compression, as it is often possible to approximate an audio
signal relatively accurately with relatively few non-zero Fourier
series coefficients. This is the basic idea behind the MP3 audio
compression algorithm\sidenote[][0em]{Yes, there is more to it. Note, that
real world compression algorithms use a lot more tricks, such as
psychoacoustics -- not storing spectral components that humans cannot
hear very well in music. These are outside the scope of this basic
course on signal processing though. However, it is already possible to
achieve significant compression just using the simple idea of
representing the signal only using the $N$ largest amplitude spectral
components. I would wager that most of the compression is achieved
with the sparse spectral representation.}.

The Python code in Listing \ref{lst:audio_compression} demonstrates
audio compression in practice. Understanding the program may require
you to be somewhat familiar with analyzing what computer programs
do. If you are just starting with programming, feel free to skip going
through this program example.

The audio compression example program relies on the discrete Fourier
transform (FFT), which we haven't covered yet. Just think of
the \verb|numpy.fft.fft| operation as the analysis step of the Fourier
series, which gives you the phase and amplitude of each frequency
component $c_k$. We'll cover the discrete Fourier transform in more
detail at a later part of this course.

\lstinputlisting[language=Python,caption={\texttt{010\_audio\_compression/audio\_compression.py}},label=lst:audio_compression]{code/010_audio_compression/audio_compression.py}

The algorithm discards weak spectral components of the signal and
stores the strong ones. This can result in significant savings in
storage space. In this example, the amount of storage required is only
5\% of the original audio signal!

