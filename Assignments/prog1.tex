\chapter{Programming Assignment 1}

Perform the following tasks. Write a report describing your results,
which is \textbf{at most two pages long}. The report is free form, but
it must be delivered in PDF format. Include your code and plots in the
report. The report should be delivered by the deadline to the
following location using your candidate ID number:
\url{http://kaira.uit.no/juha/upload/}

The Dirac comb is a periodic signal, which is defined as follows:
\begin{equation}
x(t) = \sum_{k=-\infty}^{\infty} \delta(t - k T)
\end{equation}
The signal is shown in Figure \ref{fig:dirac_comb_plot}.

\begin{marginfigure}
\begin{center}
\begin{tikzpicture}
\pgfmathdeclarefunction{p}{1}{%
  \pgfmathparse{(and(mod(#1,2)>0, mod(#1,2)<1))}%
} \begin{axis}[
width=7cm,
axis lines = center,
ymax=1.5,
ymin=0,
xmax=3.5,
xmin=-3.5,
legend pos=outer north east,
yticklabels={,,},
xticklabels={,,},
xlabel={$t$},
ylabel={$x(t)$}
]
    
    \addplot+[ycomb,color=blue,mark=triangle*,mark options={blue}] plot coordinates {
    (-3,1)
    (-2,1)
    (-1,1)
    (-0,1)
    (1,1)
    (2,1)
    (3,1)
};

\node at (axis cs:1.5,1.15) {$\displaystyle{T}$};   
\addplot [dimen] plot coordinates {(1,1.1) (2,1.1)};

    \end{axis}
\end{tikzpicture}
\end{center}
\caption{The Dirac comb signal with period $T$.}
\label{fig:dirac_comb_plot}
\end{marginfigure}


\begin{itemize}
\item[a)] Implement a program that calculates a partial sum
  approximation
  \begin{equation}
    x_N(t) = \sum_{k=-N}^{N} c_k e^{i \frac{2\pi}{T}kt}
    \label{eq:fs_pt1_eq}
\end{equation}
  of a Dirac comb with a fundamental period of $T=0.55$ seconds. Use
  $N=50$ in Equation \ref{eq:fs_pt1_eq} to define the number of
  complex exponential signals to include in the sum. Evaluate the
  signal from $t=0$ to $t=4$ seconds at 10 kHz sample rate.

  Here's some partial code, which almost does the
  job, but has several things wrong.
\begin{lstlisting}[language=Python,numbers=none]
import numpy as n
import matplotlib.pyplot as plt
# define the sample rate (Hz)
sample_rate=10000.0
# create time array 0 to 1 seconds
t=n.arange(1.0*sample_rate)/sample_rate
# initialize empty vector to hold Fourier series
sig=n.zeros(len(t),dtype=n.complex64)
N=10
# add together complex sinusoids multiplied with Fourier series coefficients
for k in range(-N,N):
    print(k)
    sig+=...# complete this line
    
plt.plot(t,sig.real)
plt.plot(t,sig.imag)
plt.xlabel("Time (s)")
plt.show()
\end{lstlisting}
If implemented correctly, the Fourier series should be real
  valued. The imaginary part should only deviate from zero by only a
  very small number due to numerical errors in finite precision
  accuracy of floating point numbers.
  
\item[b)] Figure out what modification you need to make to the Fourier series coefficients $c_k$ in order to delay the signal by 0.2 seconds.

\item[c)] Plot and verify that the coefficients obtained in b) produce the correct delay.


\end{itemize}
