\chapter{Programming Assignment 1}

The Dirac comb is a periodic signal, which is defined as follows:
\begin{equation}
x(t) = \sum_{k=-\infty}^{\infty} \delta(t - k T)
\end{equation}
The signal is shown in Figure \ref{fig:dirac_comb_plot}.

\begin{marginfigure}
\begin{center}
\begin{tikzpicture}
\pgfmathdeclarefunction{p}{1}{%
  \pgfmathparse{(and(mod(#1,2)>0, mod(#1,2)<1))}%
} \begin{axis}[
width=7cm,
axis lines = center,
ymax=1.5,
ymin=0,
xmax=3.5,
xmin=-3.5,
legend pos=outer north east,
yticklabels={,,},
xticklabels={,,},
xlabel={$t$},
ylabel={$x(t)$}
]
    
    \addplot+[ycomb,color=blue,mark=triangle*,mark options={blue}] plot coordinates {
    (-3,1)
    (-2,1)
    (-1,1)
    (-0,1)
    (1,1)
    (2,1)
    (3,1)
};

\node at (axis cs:1.5,1.15) {$\displaystyle{T}$};   
\addplot [dimen] plot coordinates {(1,1.1) (2,1.1)};

    \end{axis}
\end{tikzpicture}
\end{center}
\caption{The Dirac comb signal with period $T$.}
\label{fig:dirac_comb_plot}
\end{marginfigure}

Perform the following tasks. Write a max two-page report describing
your results. The report is otherwise free form, as long as it is in
PDF format. Include your code and plots in the report.

You will find a lot of help for this task in the lecture notes that
discusses the Fourier series. You may help each other on Perusall. It
is fine to give hints, but please try not to give away the \emph{exact
  solution}.

\begin{itemize}
\item[a)] Implement a program that calculates a partial sum
  approximation
  \begin{equation}
    x_N(t) = \sum_{k=-N}^{N} c_k e^{i \frac{2\pi}{T}kt}
\end{equation}
  of a Dirac comb with a fundamental period of $T=0.5$ seconds. Use
  $N=50$ as the number of complex exponential signals to include in
  the partial sum. Evaluate the signal from $t=0$ to $t=4$ seconds at 10
  kHz sample rate. 

  Here's some partial code, which almost does the
  job. 
\begin{lstlisting}[language=Python,numbers=none]
import numpy as n
import matplotlib.pyplot as plt
# define the sample rate (Hz)
sample_rate=10000.0
# create time array 0 to 4 seconds
t=n.arange(4.0*sample_rate)/sample_rate
# initialize empty vector to hold Fourier series
sig=n.zeros(len(t),dtype=n.complex64)
N=50
for k in range(-N,N):
    sig+=...# complete this line
    
plt.plot(t,sig.real)
plt.plot(t,sig.imag)
plt.xlabel("Time (s)")
plt.show()
\end{lstlisting}
There is a small bug in the example code, which causes the imaginary
  component of the signal \verb|sig| to not be zero valued. Find the
  bug and fix it!

\item[b)] Figure out what analytic modification you need to make to the Fourier series coefficients $c_k$ in order to delay the signal by 0.2 seconds.

\item[c)] Plot and verify that the coefficients obtained in b) produce the correct delay.

\end{itemize}