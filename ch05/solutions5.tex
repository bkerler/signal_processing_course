% Author: Jørn Olav Jensen

\newpage
\section{Suggested solutions: Elementary Signals}

\begin{enumerate}
  % Exercise 1
  \item By definition and known properties for the Dirac delta function, we have:
        \[ \int_{-\infty}^{\infty}x(t)\delta(t-\tau)dt=x(\tau). \]

        \begin{enumerate}[a)]

          % Exercise 1a)
          \item Using the above, we get:
                \[ \int_{-\infty}^{\infty}\delta(t)e^{i\omega t}dt=e^{i\omega\cdot 0}=\underline{\underline{1}}. \]

          % Exercise 1b)
          \item Similarly, we get:
                \[ \int_{-\infty}^{\infty}\delta(t-\tau)e^{i\omega t}dt=\underline{\underline{e^{i\omega\tau}}}. \]

          % Exercise 1c)
          \item If the variable is $\omega$, we have:
                \[ \int_{-\infty}^{\infty}\delta(\omega)e^{-i\omega t}d\omega=\underline{\underline{1}}, \]
                by the same reasoning as a).

          % Exercise 1d)
          \item \[ \int_{-\infty}^{\infty}\delta(\omega-\omega_{0})e^{-i\omega t}d\omega=\underline{\underline{e^{-i\omega_{0}t}}}. \]
        \end{enumerate}

  % Exercise 2
  \item Consider the integral:
        \[ \int_{-\infty}^{\infty}[u(t+L)-u(t-L)]e^{i\omega t}dt. \]
        This corresponds to a rectangular function with height $1$ and width $2L$ centered around $0$, when $L>0$. Thus, the integral can be simplified to:
        \[ \int_{-\infty}^{\infty}[u(t+L)-u(t-L)]e^{i\omega t}dt=\int_{-L}^{L}[u(t+L)-u(t-L)]e^{i\omega t}dt. \]
        The function $u(t+L)-u(t-L)\equiv 1$ on the interval $[-L, L]$, hence:
        \[ \int_{-L}^{L}[u(t+L)-u(t-L)]e^{i\omega t}dt=\int_{-L}^{L}e^{i\omega t}dt=\left[\frac{1}{i\omega}e^{i\omega t}\right]_{-L}^{L}=\frac{1}{i\omega}(e^{i\omega L}-e^{-i\omega L}). \]
        Therefore, $\alpha=\frac{1}{i\omega}$.

  % Exercise 3
  \item Using the properties of the Dirac delta function, we have:
        \[ \int_{-\infty}^{\infty}\sin(t-c)\delta(t)dt=\sin(0-c)=\sin(-c)=-\sin(c). \]
        This evaluates to $0$ for $c=\pi n$ where $n\in\mathbb{Z}$. 
        Thus, $c$ has to be $c=\pi n$.

  % Exercise 4
  \item The function $u(t)u(L-t)$ corresponds to a rectangular function of length $L$ and
        is $0$ whenever $t<0$ and $t>L$ and $1$ if $0<t<L$, therefore the integral can be written as:
        \[ \int_{-\infty}^{\infty}u(t)u(L-t)\cos(2\pi\omega t)dt=\int_{0}^{L}\cos(2\pi\omega t)dt=\frac{1}{2\pi\omega}\left[\sin(2\pi\omega t)\right]_{t=0}^{t=L}=\frac{1}{2\pi\omega}\sin(2\pi\omega L). \]

\end{enumerate}