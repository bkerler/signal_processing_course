\newpage
\section{Exercises: Complex Algebra}

\begin{enumerate}
\item Prove $e^{i\pi}+1=0$ using Euler's formula. 
\item Use Euler's formula to write $1/i$ into polar form $Ae^{i\phi}$ with $A,\phi\in\mathbb{R}$. What is the phase angle $\phi$?
\item Show that $i^{i}$ is real valued using Euler's formula. Use Python to calculate this value and verify that it is real. 
\item Use Euler's formula to show that de Moivre's formula is valid for $n\in\mathbb{Z}$:
$$[\cos(x)+i\sin(x)]^{n}=\cos(nx)+i\sin(nx)$$

\item Using Euler's formula, it is possible to determine the $n$th root of unity. What this means is that we look for all unique values of $z$ which satisfy the following equation where $n$ is a positive integer:
\begin{equation}
    z^{n}=1.
    \label{ch03:eq_unity}
\end{equation}
You probably already know the answer in the case of $n=2$, for which there are two solutions: $z_{0}=1$ and $z_{1}=-1$ as both $1^{2}=1$ and $(-1)^{2}=1$. \\
Provide a general formula that gives $n$ unique values of $z$ that
satisfy Equation \ref{ch03:eq_unity}. How many unique solutions of $z$ are there
for each value of $n$? Hint: remember that $e^{i 2\pi k} = 1$ where
$k$ is an integer. 
\item Consider the equation $z^{5}=1$, where $z\in\mathbb{C}$. 
\begin{itemize}
    \item[a)] Find the five solutions $w_{k}$ for $k=0,1,2,3,4$, to the equation $z^{5}=1$. 
    \item[b)] Create a Python program to plot the solutions to this equation in the complex plane. Add lines to your plot which connect the point $w_{k}$ to $w_{k-1}$ mod $5$. What is the shape drawn?
    \item[c)] Add the unit circle to the plot you found in b). 
\end{itemize}

\item Use the inverse Euler formula to convert $(1-i) e^{-i \omega t} + (1+i) e^{i
  \omega t}$ into the following form:
\begin{equation*}
    A \cos(\omega t  + \phi)
\end{equation*}
with $A\in \mathbb{R}$. What is $A$ and what is $\phi$?

\item Prove using Euler's formula that:
\begin{equation*}
    \cos(3\theta)= \cos^3(\theta)-3\cos(\theta)\sin^2(\theta).
\end{equation*}

\item Use Euler's formula to prove the following trigonometric identity:
\begin{align*}
\cos(\alpha + \beta)&=\cos(\alpha)\cos(\beta) - \sin(\alpha)\sin(\beta)\,\,.
\end{align*}

\item Another definition of $e^{z}$ (the complex exponential) is by an extension of the Taylor series expansion of $e^{x}$ (the real exponential), as follows
$$e^{z}:=\sum_{k=0}^{\infty}\frac{z^{k}}{k!}, \quad\quad \text{for}\ z\in\mathbb{C}.$$
That is, $e^{z}$ is taken to mean this infinite series. Use this Taylor series expansion definition to show that
\begin{equation}
  e^{i\theta} = \cos(\theta) + i\sin(\theta).
  \label{ch03:euler_formula}
\end{equation}
To do this, compute the Taylor series expansion of $e^{i\theta}$. 

\end{enumerate}