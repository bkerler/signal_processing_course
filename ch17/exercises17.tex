\newpage
\section{Exercises: Arbitrary Frequency Response Filters}

\begin{enumerate}

    % Exercise 1
    \item The Python code in Listing \ref{lst:fftex} creates a signal consisting of narrowband sinusoids with large amplitudes ($x_1[n]$), and three time shifted unit impulses with weak amplitudes ($x_2[n]$):
          \begin{equation}
              x[n] = x_1[n] + x_2[n]
          \end{equation}

          \lstinputlisting[language=Python, caption={\texttt{fftex.py}}, label=lst:fftex]{ch17/code/fftex.py}

          \begin{enumerate}[a)]
              % Exercise 1a)
              \item Estimate the spectrum of the signal $x[n]$ using a Hann window $w[n]$ of length $N=16384$ to reduce spectral leakage.
                    \begin{equation}
                        \hat{x}_w[k]= \sum_{n=0}^{N-1}w[n]x[n]e^{-i\frac{2\pi}{N}nk}
                    \end{equation}
                    Use FFT to evaluate the DFT.

                    Make a plot of the magnitude spectrum with power in dB scale
                    $10\log_{10}|\hat{x}_w[\hat{\omega}_k]|^2$. Only plot the positive
                    frequencies between 0 and $\pi$ with units of radians per
                    sample. Identify the frequency ranges that are occupied by strong
                    narrowband spectral components.

                    % Exercise 1b)
              \item Filter out the large magnitude frequency components in
                    frequency domain and inverse DFT to obtain a time domain
                    representation of the signal.
                    \begin{equation}
                        y[n]= \mathcal{F}_D^{-1}\left\{ \hat{h}[k]\hat{x}_w[k] \right\}
                    \end{equation}
                    Use FFT to evaluate the inverse DFT. Your plot should look like
                    Figure \ref{fig:filtered_weak_signal}.

                    % Exercise 1c)
              \item Explain why the filtered signal $y[n]$ still looks like the
                    original weak signal $x_2[n]$ consisting of unit impulses, even
                    though it is missing some frequency components.
          \end{enumerate}

\end{enumerate}
