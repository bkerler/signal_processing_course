\newpage
\section{Exercises: Ideal and Tapered Filters}

\begin{enumerate}
\item Define the ideal band-stop filter as: \footnote{that is $\mathcal{H}_{\mathrm{BS}}(\hat{\omega})=1 - \mathcal{H}_{\mathrm{BP}}(\hat{\omega})$.}
\begin{equation}
\mathcal{H}_{\mathrm{BS}}(\hat{\omega}) = \left\{ \begin{array}{cc}
0 & \hat{\omega}_0 < |\hat{\omega}| < \hat{\omega}_1 \\
1 & \mathrm{otherwise}
\end{array}\right.\,\,.
\end{equation}
Derive the impulse response for the band-stop filter.

\item Consider the band-stop filter you found in the previous exercise. 
The goal of this exercise is to implement this filter in Python and apply it to filter an audio file.

\begin{enumerate}[a)]
\item Let $w[n]$ be a tapered window function, like the Hann window of length $N$. Show that the windowed filter can be written as:
$$h_{w}[n]=\delta[n-N/2]w[n] +  \frac{\sin(\hat{\omega}_{0}(n-N/2))}{(n-N/2)\pi}w[n] - \frac{\sin(\hat{\omega}_{1}(n-N/2))}{(n-N/2)\pi}w[n].$$

\item Download the \verb|crappy.wav| file from the link:
\begin{center}
\verb|https://bit.ly/3CvE89s| 
\end{center}
The audio file contains a guitar playing, but it is impossible to hear due to noise. 
Your task is to filter out this noise using a band-stop filter. 
The noise in the signal is contained in a frequency band between $f_{0}=3$ kHz and $f_{1}=5$ kHz, 
but try to experiment with these to see what happens. Implement a band-stop filter using 
Python to remove the noise and recover the original audio. You can use the code shown in Listing \ref{bandex} 
as a starting point. 
\end{enumerate}

\lstinputlisting[language=Python,caption=Band-stop filter starting point,label=bandex]{ch13/code/ex13_2_precode.py}

\end{enumerate}
