\newpage
\section{Exercises: Ideal and Tapered Filters}

\begin{enumerate}
\item Define the ideal band-stop filter as: \footnote{that is $\mathcal{H}_{\mathrm{BS}}(\hat{\omega})=1 - \mathcal{H}_{\mathrm{BP}}(\hat{\omega})$.}
\begin{equation}
\mathcal{H}_{\mathrm{BS}}(\hat{\omega}) = \left\{ \begin{array}{cc}
0 & \hat{\omega}_0 < |\hat{\omega}| < \hat{\omega}_1 \\
1 & \mathrm{otherwise}
\end{array}\right.\,\,.
\end{equation}
Derive the impulse response for the band-stop filter.

\item Consider the band-stop filter you found in the previous exercise. The goal of this exercise is to implement this filter in Python and apply it to filter audio files.

\begin{enumerate}[a)]
\item Let $w[n]$ be a tapered window function, like the Hann window of length $N$. Show that the windowed filter can be written as:
$$h_{w}[n]=\delta[n-N/2]w[n] +  \frac{\sin(\hat{\omega}_{0}(n-N/2))}{(n-N/2)\pi}w[n] - \frac{\sin(\hat{\omega}_{1}(n-N/2))}{(n-N/2)\pi}w[n].$$

\item Implement the filter above using Python. You can use the code in Listing \ref{band_stop} as a starting point. 
\footnote{Don't worry if you can't understand everything in the code yet, we'll get to it later when we introduce the discrete Fourier transform and the fast Fourier transform algorithm.}
\lstinputlisting[language=Python,caption=Precode for exercise 2,label=band_stop]{ch13/code/ex13_2_precode.py}

\item Apply the filter to the test signal. The test signal has three frequency components, these being $f_{0}=196$, $f_{1}=2f_{0}$ and $f_{2}=5f_{0}$. 
Use the stop-band filter to filter out $f_{0}$ and $f_{1}$. Run your implementation and play the audio file filtered.wav. 
Does it sound correct? Try using adjusting the frequencies and apply your filter on other pieces of music. What happens? 
\footnote{Ear warning! If you do something wrong, the audio might be very loud or sound horrible, so be careful.}

\end{enumerate}






\end{enumerate}
