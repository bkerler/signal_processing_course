\newpage
\section{Exercises: Signals and Systems}
\begin{enumerate}

\item Consider the following signal $V(t)$, which describes the electric potential of a circuit as a function of time. The instantaneous power flowing through this circuit is given by the following equation:
\begin{equation*}
    P(t)=R^{-1}[V(t)]^2.
\end{equation*}
It is possible to consider the function $R^{-1}[\cdot]^2$ that converts electric potential into power as a system in signal processing terminology.
\begin{enumerate}[a)]
\item Is this system linear?
\item Is this system time invariant?
\end{enumerate}
Prove your result.

\item A discrete-time system is defined as:
\begin{equation}
  y[n]= \frac{1}{5}\sum_{k=0}^{4} x[n-k].
\end{equation}
  \begin{enumerate}[a)]
  \item Explain in words what this system does to the input signal $x[n]$.
  \item Is this system linear?
  \item Is this system time-invariant?
  \end{enumerate}
Use the tests for linearity and time-invariance to justify your result. 

\item A time scaling system adjusts the scaling of the independent variable: $y(t) = x(\alpha t)$,
when $x(t)$ is the signal fed into the system and $y(t)$ is the output.

\begin{enumerate}[a)]
\item What is the effect on the signal when $0<\alpha<1$?
\item What about $\alpha>1$?
\item Is this system linear?
\item time-invariant?
\end{enumerate}

\item Prove that the time derivative operator $y(t) = \mathcal{T}\{x(t)\} = \frac{d}{dt} x(t)$
is a linear time-invariant system.

\item The guitar amplifier system given in Equation \ref{clipamp} is not a linear system. Prove this. Hint: you can use proof by example, by finding a case where the requirement of linearity is not met. 

\item Is the guitar amplifier system given in Equation \ref{clipamp} a time-invariant system? Prove your result by using the formal test for time-invariance.

\item The code in Listing \ref{lst:audio} implements a linear amplifier system for an audio signal:
\begin{equation}
y(t) = \alpha x(t)
\end{equation}
where the input signal is $x(t)$, the output signal is $y(t)$, and the amplification is $\alpha$. You can find the code and audio file on Github.
\lstinputlisting[language=Python,caption={\texttt{003\_guitar/amplifier.py}},label=lst:audio]{code/003_guitar/amplifier.py}
Run this code and verify that the audio signal stored in
file \verb|guitar_amp.wav| is amplified. You will need to do this by inspecting a before and after amplification plot of the audio signal, as the example code will normalize the amplitude of the signal before storing it to file.

You can use this script as a basis for experimenting with audio signal processing later on during this course.

\item Change the code in Listing \ref{lst:audio} is such a way that it implements a clipping amplifier as shown in Equation: \ref{clipamp}. Figure out suitable values for $\alpha$ and $\beta$ that make the guitar sound distorted.

\end{enumerate}
