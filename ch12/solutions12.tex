\newpage
\section{Suggested solutions: Discrete-time Fourier Transform (DTFT)}
\begin{enumerate}
\item Suppose $x[n]$ is a real discrete-time signal, then by definition
$$\hat{x}(\hat{\omega})=\sum_{n=-\infty}^{\infty}x[n]e^{-i\hat{\omega}n}.$$
Apply complex conjugation to obtain
$$\hat{x}^{*}(\hat{\omega})=\sum_{n=-\infty}^{\infty}(x[n]e^{-i\hat{\omega}n})^{*}=\sum_{n=-\infty}^{\infty}x[n]^{*}e^{i\hat{\omega}n}.$$
Since the signal $x[n]$ is real, we obtain $x^{*}[n]=x[n]$. Then evaluating at $\hat{x}^{*}(-\hat{\omega})$ is
$$\hat{x}^{*}(-\hat{\omega})=\sum_{n=-\infty}^{\infty}x[n]e^{-i\hat{\omega}n}$$
which is equal to $\hat{x}(\hat{\omega})$. 

\item Suppose $\hat{x}(\hat{\omega})$ is a real signal, then consider
$$x[n]=\frac{1}{2\pi}\int_{-\pi}^{\pi}\hat{x}(\hat{\omega})e^{i\hat{\omega}n}d\hat{\omega},$$
thus
$$\hat{x}^{*}[n]=\frac{1}{2\pi}\int_{-\pi}^{\pi}(\hat{x}(\hat{\omega})e^{i\hat{\omega}n})^{*}d\hat{\omega}=\frac{1}{2\pi}\int_{-\pi}^{\pi}\hat{x}(\hat{\omega})^{*}e^{-i\hat{\omega}n}d\hat{\omega}$$
and since $\hat{x}(\hat{\omega})$ is real we obtain $\hat{x}^{*}(\hat{\omega})=\hat{x}(\hat{\omega})$, then evaluating $\hat{x}^{*}[n]$ at $-n$ gives
$$\hat{x}^{*}[-n]=\frac{1}{2\pi}\int_{-\pi}^{\pi}\hat{x}(\hat{\omega})e^{i\hat{\omega}n}d\hat{\omega},$$
which is equal to $x[n]$. 

\item Let $\hat{x}(\hat{\omega})=1$, then the inverse discrete-time Fourier transform is
$$x[n]=\frac{1}{2\pi}\int_{-\pi}^{\pi}\hat{x}(\hat{\omega})e^{i\hat{\omega}n}d\hat{\omega}=\frac{1}{2\pi}\int_{-\pi}^{\pi}e^{i\hat{\omega}n}d\hat{\omega}=\frac{1}{2\pi}\left[\frac{1}{in}e^{i\hat{\omega}n}\right]_{-\pi}^{\pi}$$
which gives
$$x[n]=\frac{1}{2\pi in}(e^{i\pi n}-e^{-i\pi n})=\frac{1}{\pi n}\sin(\pi n).$$
If $n\neq 0$, then this is $0$. If $n=0$ we get, using L'Hôpital's rule
$$\lim_{n\to 0}\frac{\sin(\pi n)}{\pi n}=1.$$
Hence this corresponds to a Dirac-delta $\delta[n]$, so by linearity we get $42i\delta[n]$. 

\item Consider three systems
\begin{align*}
    \mathcal{T}_{1}\{x[n]\}&=x[n]-x[n-1], \\
    \mathcal{T}_{2}\{x[n]\}&=x[n]+x[n-2], \\
    \mathcal{T}_{3}\{x[n]\}&=x[n-1]+x[n-2].
\end{align*}

\begin{enumerate}[a)]
\item The impulse responses are given as $h_{i}[n]=\mathcal{T}_{i}\{\delta[n]\}$ for $i=1,2,3$. We get
\begin{align*}
    h_{1}[n]&=\delta[n]-\delta[n-1], \\
    h_{2}[n]&=\delta[n]+\delta[n-2], \\
    h_{3}[n]&=\delta[n-1]+\delta[n-2].
\end{align*}

\item In time domain we have 
$$y[n]=\mathcal{T}_{1}\{\mathcal{T}_{2}\{\mathcal{T}_{3}\{x[n]\}\}\}$$
which can be written as
$$y[n]=h[n]*x[n]$$
since this an LTI system, where 
$$h[n]=h_{1}[n]*h_{2}[n]*h_{3}[n].$$
In frequency domain, we have
$$\mathcal{H}(\hat{\omega})=\mathcal{H}_{1}(\hat{\omega})\mathcal{H}_{2}(\hat{\omega})\mathcal{H}_{3}(\hat{\omega}).$$
Each $\mathcal{H}_{1}$ is found by the discrete-time Fourier transform:
\begin{align*}
    \mathcal{H}_{1}(\hat{\omega})&=1-e^{-i\hat{\omega}}, \\
    \mathcal{H}_{2}(\hat{\omega})&=1+e^{-2i\hat{\omega}}, \\
    \mathcal{H}_{3}(\hat{\omega})&=e^{-i\hat{\omega}}+e^{-2i\hat{\omega}}.
\end{align*}
Then
$$\mathcal{H}(\hat{\omega})=(1-e^{-i\hat{\omega}})(1+e^{-2i\hat{\omega}})(e^{-i\hat{\omega}}+e^{-2i\hat{\omega}})=e^{-i\hat{\omega}}-e^{-5i\hat{\omega}},$$
as we wanted to show. 
\end{enumerate}

\item

\begin{enumerate}[a)]
\item If $\hat{x}(\hat{\omega})=1-2e^{3i\hat{\omega}}$, then the IDTFT is
$$x[n]=\delta[n]-2\delta[n+3].$$

\item If $\hat{x}(\hat{\omega})=2e^{-3i\hat{\omega}}\cos(\hat{\omega})$ then by rewriting the function as follows
$$\hat{x}(\hat{\omega})=2e^{-3i\hat{\omega}}\frac{1}{2}(e^{i\hat{\omega}}+e^{-i\hat{\omega}})=e^{-3i\hat{\omega}}(e^{i\hat{\omega}}+e^{-i\hat{\omega}})=e^{-2i\hat{\omega}}+e^{-4i\hat{\omega}},$$
which gives that
$$x[n]=\delta[n-2]+\delta[n-4].$$

\item For the function $\hat{x}(\hat{\omega})=e^{-i42\hat{\omega}}\frac{\sin(10.5\hat{\omega})}{\sin(\hat{\omega}/2)}$, we notice that $\frac{\sin(10.5\hat{\omega})}{\sin(\hat{\omega}/2)}$ is a Dirichlet kernel with $L=21$. Therefore, it can be rewritten as
$$\frac{\sin(10.5\hat{\omega})}{\sin(\hat{\omega}/2)}=\sum_{k=0}^{20}e^{-i\hat{\omega}k},$$
giving
$$\hat{x}(\hat{\omega})=e^{-i42\hat{\omega}}\sum_{k=0}^{20}e^{-i\hat{\omega}k}=\sum_{k=0}^{20}e^{-i\hat{\omega}(k+42)}=\sum_{k=42}^{62}e^{-i\hat{\omega}k}.$$
The IDTFT is then
$$x[n]=\frac{1}{2\pi}\int_{-\pi}^{\pi}\sum_{k=42}^{62}e^{-i\hat{\omega}k}e^{i\hat{\omega}n}d\hat{\omega} =\sum_{k=42}^{62}\delta[n-k].$$
\end{enumerate}









\end{enumerate}